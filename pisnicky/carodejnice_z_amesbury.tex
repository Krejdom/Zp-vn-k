\section*{\Huge ČARODĚJNICE Z AMESBURY}
\emph{Asonance}\\

Zuzana \chord{Dmi}byla dívka, \chord{C}která žila v \chord{Dmi}Amesbury,\\
s jasnýma \chord{F}očima a \chord{C}řečmi pánům \chord{Dmi}navzdory,\\
souse\chord{F}dé o ní \chord{C}říkali, že \chord{Dmi}temná kouzla \chord{Ami}zná\\
a \chord{B}že se lidem \chord{Ami}vyhýbá a s \chord{B}ďáblem \chord{C}pletky \chord{Dmi (G)}má.\\

\begin{large}

Onoho léta náhle mor dobytek zachvátil\\
a pověrčivý lid se na pastora obrátil,\\
že znají tu moc nečistou, jež krávy zabíjí,\\
a odkud ta moc vychází, to každý dobře ví.\\

Tak Zuzanu hned před tribunál předvést nechali,\\
a když ji vedli městem, všichni kolem volali:\\
"Už konec je s tvým řáděním, už nám neuškodíš,\\
teď na své cestě poslední do pekla poletíš!"\\

Dosvědčil jeden sedlák, že zná její umění,\\
ďábelským kouzlem prý se v netopýra promění\\
a v noci nad krajinou létává pod černou oblohou,\\
sedlákům krávy zabíjí tou mocí čarovnou.\\

Jiný zas na kříž přísahal, že její kouzla zná,\\
v noci se v černou kočku mění dívka líbezná,\\
je třeba jednou provždy ukončit ďábelské řádění,\\
a všichni křičeli jako posedlí:"Na šibenici s ní!"\\

Spektrální důkazy pečlivě byly zváženy,\\
pak z tribunálu povstal starý soudce vážený:\\
"Je přece v knize psáno: nenecháš čarodějnici žít\\
a před ďáblovým učením budeš se na pozoru mít!"\\

Zuzana stála krásná s hlavou hrdě vztyčenou\\
a její slova zněla klenbou s tichou ozvěnou:\\
"Pohrdám vámi, neznáte nic nežli samou lež a klam,\\
pro tvrdost vašich srdcí jen, jen pro ni umírám!"\\

Tak vzali Zuzanu na kopec pod šibenici\\
a všude kolem ní se sběhly davy běsnící,\\
a ona stála bezbranná, však s hlavou vztyčenou,\\
zemřela tiše samotná pod letní oblohou.

\end{large}

\newpage
