\section*{\Huge PRVNÍ DEN\\NA FAKULTĚ INFORMATIKY}
\emph{Jan Pater, FI MU}\\

Bylo pondělí dvacátého září,\\
já jsem kráčel Botanickou ulicí.\\
Kráčel se širokým úsměvem na tváři,\\
očekávajíce kupu znalostí.\\
\\
Na recepci v mapě školy marně hledám,\\
kde to stojím, kde jsem to sakra skys?\\
Přidávám se k davu, mířím směr D1,\\
větší chaos už je snad jenom ten IS.\\
\\
\textregistered:
\emph{To jsem si nemyslel, že to bude tak složitý,\\
přežít první den na Fakultě informatiky.}\\
\\
Na informatice vítá mě pan Hliněný,\\
nevím co jsem si z tý přednášky odnes.\\
Student A se studentem B v relaci není,\\
ale to foukání z beden to byl děs!\\
\\
Matika byla snad lepší plus-mínus,\\
pan Panák přišel mírně zmatený.\\
\uv{Prosím vás, nemá tam být náhodou arkus sinus?}\\
\uv{To přece výsledek vůbec neovlivní!}
\hspace{1cm}\textregistered\\
\\
Po přednášce vyrazil jsem do menzy,\\
pizza za čtyřicet -- no to by se dalo!\\
Paní kuchařka však měla ňáký hemzy\\
a věta zabolela: \uv{Máte tam málo.}\\

\newpage
Z další přednášky si toho moc nepamatuju,\\
přišlo mi to nějak celé zabité.\\
Příště už hraním her nikdy nezabíjím nudu\\
a už vůbec na Úvodu do IT.
\hspace{1cm}\textregistered\\
\\
Však s další přednáškou zhroutil se mi svět.\\
Co to je? To, to snad není normální!\\
Teror, hrůza, bída, strach, tam a zpět.\\
Funkcionální programování.
\hspace{1cm}\textregistered\\
\\
Po přednášce potom protřel jsem si voči,\\
svou budoucnost nevidím růžovou.\\
Vsadím se, že ještě než ten semestr skončí,\\
já nafasuju vestu oranžovou.\\
\\
Prý to ale zase taková hrůza není,\\
no já byl po tom dni fakt celý zbitý.\\
Jak že bylo to jedno český přísloví?\\
No přece: \uv{Bez práce nejsou kredity.}\\
\\
\textregistered:
\emph{To jsem si nemyslel, že to bude tak složitý,\\
ale přežil jsem první den na Fakultě informatiky.}

\newpage
